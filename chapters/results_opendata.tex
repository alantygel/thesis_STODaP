\chapter{Results of Open Data Research}
\label{chap:dl_results}
%\begin{table}[htb]
%\ABNTEXfontereduzida
%\caption[Níveis de investigação]{Níveis de investigação.}
%\label{tab-nivinv}
%\begin{tabular}{p{2.6cm}|p{6.0cm}|p{2.25cm}|p{3.40cm}}
%  %\hline
%   \textbf{Nível de Investigação} & \textbf{Insumos}  & \textbf{Sistemas de Investigação}  & \textbf{Produtos}  \\
%    \hline
%    Meta-nível & Filosofia\index{filosofia} da Ciência  & Epistemologia &
%    Paradigma  \\
%    \hline
%    Nível do objeto & Paradigmas do metanível e evidências do nível inferior &
%    Ciência  & Teorias e modelos \\
%    \hline
%    Nível inferior & Modelos e métodos do nível do objeto e problemas do nível inferior & Prática & Solução de problemas  \\
%   % \hline
%\end{tabular}
%\legend{Fonte: \citeonline{van86}}
%\end{table}


%Já a \autoref{tabela-ibge} apresenta uma tabela criada conforme o padrão do
%\citeonline{ibge1993} requerido pelas normas da ABNT para documentos técnicos e
%acadêmicos.

%\begin{table}[htb]
%\IBGEtab{%
%  \caption{Um Exemplo de tabela alinhada que pode ser longa
%  ou curta, conforme padrão IBGE.}%
%  \label{tabela-ibge}
%}{%
%  \begin{tabular}{ccc}
%  \toprule
%   Nome & Nascimento & Documento \\
%  \midrule \midrule
%   Maria da Silva & 11/11/1111 & 111.111.111-11 \\
%  \midrule 
%   João Souza & 11/11/2111 & 211.111.111-11 \\
%  \midrule 
%   Laura Vicuña & 05/04/1891 & 3111.111.111-11 \\
%  \bottomrule
%\end{tabular}%
%}{%
%  \fonte{Produzido pelos autores.}%
%  \nota{Esta é uma nota, que diz que os dados são baseados na
%  regressão linear.}%
%  \nota[Anotações]{Uma anotação adicional, que pode ser seguida de várias
%  outras.}%
%  }
%\end{table}



\begin{table}[]
\ABNTEXfontereduzida
\caption[Motivations, Impediments and Improvements.]{Motivations, Impediments and Improvements indicated in answers to Question 4.}
\label{tab:dl_results1}
\begin{tabular}{|p{1cm}|p{4.4cm}|p{4.4cm}|p{4.4cm}|}
%\begin{tabular}{|l|l|l|l|}
\hline
\multicolumn{4}{|l|}{Question 4: Why have you attended to the course? Why do you think open data is important?} \\ \hline
\multicolumn{1}{|c|}{\textbf{\#}} & \multicolumn{1}{c|}{\textbf{Motivations}}                                                                       & \multicolumn{1}{c|}{\textbf{Impediments}}                                                                                                               & \multicolumn{1}{c|}{\textbf{Improvements}} \\ \hline
4.1&Work with data and link different information to create arguments&There is a mismatch between amount of data released and the capacity of social movements to analyse it &Make investments in education for open data use\\ \hline
4.2&Be able to work with data driven journalism&There are many barriers to access information&Promote publicity about existence of data\\ \hline
4.3&Use data to denounce injustices&Open Data is unknown for most social movements&Improve knowledge about how to search for data\\ \hline
4.4&Data can give basis to stimulate new claims &There is no full transparency in government actions&Enable access to information, without discrimination\\ \hline
4.5&Translate data into information for readers&Most of the people have little informatics ability&\\ \hline
4.6&Produce data in juridical research&&\\ \hline
4.7&Open data can stimulate analysis&&\\ \hline
4.8&Open data can stimulate new data&&\\ \hline
4.9&Validate/legitimate arguments in communication with data&&\\ \hline
4.10&Use data to understand the capitalist society&&\\ \hline
4.11&Understand the resistances against oppression with data&&\\ \hline
4.12&Fight corruption using spending data &&\\ \hline
4.13&Make better use of information, a central point in class conflicts&&\\ \hline
4.14&Unveil data manipulation&&\\ \hline
\end{tabular}
\legend{Source: \citeonline{Tygel2015a}}
\end{table}


\begin{table}[]
\ABNTEXfontereduzida
\caption{Impediments pointed in answers to Question 8.}
\label{tab:dl_results2}
\begin{tabular}{|l|l|}
\hline
\multicolumn{2}{|l|}{Question 8: What is the main impediment perceived by using data?}                                       \\ \hline
\multicolumn{1}{|c|}{\textbf{\#}} & \multicolumn{1}{c|}{\textbf{Impediments}}                                                \\ \hline
8.1                               & The lack of knowledge about data production process makes interpretation difficult       \\ \hline
8.2                               & It is hard to understand data connection and linking possibilities                       \\ \hline
8.3                               & Finding data in the web is hard /Open data portals are complicated                       \\ \hline
8.4                               & Access to data outside the web is hard / FoIA application is complicated                 \\ \hline
8.5                               & Data organization is confused                                                            \\ \hline
8.6                               & Data formats does not help its use                                                       \\ \hline
8.7                               & The state presents data through different platforms which increase the need for training \\ \hline
8.8                               & The need of specific software tools makes data usage harder                              \\ \hline
8.9                               & Some important data is concealed                                                         \\ \hline
8.10                              & Most data is outdated                                                                    \\ \hline
8.11                              & The querying interfaces present too much information                                     \\ \hline
8.12                              & Access to raw data is hard                                                               \\ \hline
8.13                              & Government agencies do not follow common data standards                                  \\ \hline
8.14                              & "Data interpretation is difficult                                                        \\ \hline
8.15                              & Linking data from different sources is difficult without appropriate tools and metadata  \\ \hline
\end{tabular}
\legend{Source: \citeonline{Tygel2015a}}
\end{table}


\begin{table}[]
\ABNTEXfontereduzida
\caption{Improvements indicated in answers to Question 9.}
\label{tab:dl_results3}
\begin{tabular}{|l|l|}
\hline
\multicolumn{2}{|l|}{Question 9: How do you imagine that the use of data could be improved?}           \\ \hline
\multicolumn{1}{|c|}{\textbf{\#}} & \multicolumn{1}{c|}{\textbf{Improvements}}                         \\ \hline
9.1                               & Provide user-friendly interfaces                                   \\ \hline
9.2                               & Provide education on statistics/mathematics                        \\ \hline
9.3                               & Standardize open government data                                   \\ \hline
9.4                               & Provide user-friendly language (avoid technical terms)             \\ \hline
9.5                               & Provide wider training possibilities                               \\ \hline
9.6                               & Promote more advertising of open government data initiatives       \\ \hline
9.7                               & Promote more advertising of social movements open data initiatives \\ \hline
9.8                               & Foster more research on open data and social movements             \\ \hline
9.9                               & Improve data search engines                                        \\ \hline
9.10                              & Increase the offer of open data sources                            \\ \hline
9.11                              & Avoid the need of intermediaries for data interpretation           \\ \hline
9.12                              & Improve open data portals                                          \\ \hline
\end{tabular}
\legend{Source: \citeonline{Tygel2015a}}
\end{table}

