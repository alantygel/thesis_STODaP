%\begin{englishabstract}{Colocar aqui o título da dissertação em inglês}{coloque aqui as palavras-chave em inglês, separadas por vírgula}
%Avançando no Uso de Dados Abertos Através do Enriquecimento Semânticos de Metadados e da Alfabetização em Dados

\begin{resumo}[Abstract]
 \begin{otherlanguage*}{english}
The extensive publishing of data in open formats on the Web seems to be an irreversible tendency.
Regarding governments, claims for more transparency coming from the civil society are forcing public administrations to ease the access to government data, mostly through central data repositories.
These data repositories, or Open Data Portals (ODPs), have the objective of being a meeting point between government and citizens willing to access public data.
Hence, it is expected a greater transparency of public administrations, which in turn feed democracy with a well informed population, and inhibits corruption through the possibility of open scrutiny by the public.
Alongside the great expectations created by the open data policies, we also verify a wide range of problems which still hinder a more effective growing of the open data initiatives.
During the research related to this thesis, two problems called the attention: (i) the lack of conceptual organization of the open data portals, and (ii) the difficulties of the general public for dealing with data.
Thus, this thesis expects to bring a contribution for the field of open data by proposing approaches for both problems.
ODP managers use several types of metadata to organize the datasets, one of the most important ones being the tags.
However, the tagging process is subject to many problems, such as synonyms, ambiguity or incoherence, among others.
As our empiric analysis of ODPs shows, these issues are currently prevalent in most ODPs and effectively hinders the reuse of Open Data.
In order to address these problems, we develop and implement an approach for tag reconciliation in Open Data Portals, encompassing local actions related to individual portals, and global actions for adding a semantic metadata layer above individual portals.
Several studies attest that, on the open data policies adopted by governments, there is a greater attention on data publishing, while enhancing the capacity of people for using these data remains in background. 
Thus, even if data are published, it is necessary to have an empowered society to deal with it.
Otherwise, there is a risk of creating an elite able to profit from these information, deepening even more the digital divide, especially in countries like Brazil.
In order to tackle this matter, we present in this thesis an approach for data literacy, inspired in the pedagogy of popular education and in the participatory action-research technique.
It is expected that this thesis contribute with and advance in the democratisation of information, contextualizing in a more adequate form the publication of open data, and allowing its use by a broader part of the population.


\textbf{Keywords}: Open Data. Open Data Portal. Metadata Reconciliaton. Semantic Lifting. Data Literacy.
 \end{otherlanguage*}
\end{resumo}


