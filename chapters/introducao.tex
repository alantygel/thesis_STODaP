\chapter{Introduction}

This thesis is essentially about open data.
Given the growing importance of the topic, open data is discussed through several points of views.
Although looking at it from a Computer Science perspective, it was not possible to skip political, social and economical aspects while discussing the topic.
This work should thus be regarded as a multidisciplinary effort to contribute to a field that is heavily related to Computer Science, but far from being restricted to it.

In this introductory chapter, some motivations behind the topic of open data are briefly exposed, highlighting selected problems observed in the literature.
The hypothesis from where this thesis starts are posed, as well as the objectives that we aim to achieve with this work.
We further explain the methodology used in order to develop this thesis, and finally describe the structure of the rest of this text.

\section{Motivation}
% * crescimento do open data
Current numbers about the open data scene leave no doubt about the central importance of this topic in contemporary society.
The \href{http://index.okfn.org/place/}{Open Data Index} monitored in 2015 open datasets published by 122 countries all over the world, on topics related to budget, national statistics, procurements, maps and many others.
Regarding the European landscape, the \href{http://opendatamonitor.eu/}{Open Data Monitor} counts 173 open data catalogues in the continent, which sums an amount of 1472 GB of data.

The movement towards opening datasets has its roots related to a series of access to information laws. 
According to the \href{http://right2info.org}{right2info.org} platform, these are laws that ``establish the right and procedures for the public to request and receive government-held information''.
\citeonline{Vleugels2012} presents a comprehensive list of 273 Freedom of Information Acts (FOIA), being 93 of national, 180 of sub-national and 3 of international scope.
Even though the first occurrence of this kind of law dates from 1766, in Sweden, the vast majority of them were created after the year 2000.

It is no coincidence that 9 years later, United States and United Kingdom launched their Open Data Portals (ODPs), a one-stop-shop for publishing and consuming government data.
Nowadays, several countries already implemented their ODPs, together with numerous states and municipalities.
Universities and research centres are also joining strategies for putting data available on the Web.
More than 1600 ODPs were surveyed by \href{https://www.opendatasoft.com}{OpenDataSoft}.
The potential of changing the very basis of democratic processes took the United Nations (UN) to coin the term \emph{Data Revolution} to designate ``the new world of data, a world in which data are bigger, faster and more detailed than ever before''~\cite[p.4]{DataRevolutionGroup2014}.
Still according to the UN, these data is ``creating unprecedented possibilities for informing and transforming society and protecting the environment''~\cite[p.4]{DataRevolutionGroup2014}.

While the numbers about open data scenario can generate an enthusiastic hope that this movement will solve many problems of the society, there are also critique voices claiming that the promises are still far from being realised, and also that there are some hidden threats that should be alerted.

\section{Not Only Advantages -- Open Data Impediments}

Despite its recent popularity, Open Data and Open Data Portals still face significant problems.
\citeonline{Zuiderwijk2012} collected 118 socio-technical impediments for use of open data from interviews, workshops and literature.
Some cited impediments were ``absence of commonly agreed metadata'', ``insufficiency of metadata'', ``the lack of interoperability'' and ``difficulty in searching and browsing data'', showing that a great challenge for ODPs is the organization of data.

The open data organization challenge can be subdivided into two aspects: 1) structuring and organizing the datasets themselves and 2) providing well-structured and organized metadata for the datasets.
The first aspect was, for example, tackled by approaches for semantic lifting of data by~\citeonline{DBLP:conf/i-semantics/ErmilovAS13} and~\citeonline{Ding2011325}, who tried to build general strategies for putting large open government datasets in the Link Open Data (LOD) cloud.
For the standardized structuring metadata, the Data Catalog Vocabulary (DCAT)\footnote{Available at \url{http://www.w3.org/TR/vocab-dcat/}}~\cite{conf/i-semantics/CyganiakMP10} was developed.
However, the cross-portal metadata alignment and reconciliation cannot be addressed by DCAT.

Besides the data organization challenge, another very much cited impediment to the use of open data is the lack of capacity of individuals and groups for dealing with open data.
There has been a recently growing consensus on defining the skills of consuming, publishing and understanding data under the concept of \emph{Data Literacy}.

As observed by~\citeonline{Bhargava2015}, one of the first mentions of the term \emph{Data Literacy} called the attention for its importance on the context of evaluation of information, together with  Information Literacy and Statistical Literacy. 
In 2004, Schield reinforced the importance of teaching these three literacies for ``students who need to critically evaluate information in arguments'' \cite[p.1]{Schield2004}.

Although not mentioning directly the term Data Literacy, the previously cited collection of open data impediments~\cite{Zuiderwijk2012} dedicates a section for problems related to \emph{understand ability}.
Among them we find, for example, ``Lack of skills and capabilities to use the data and'' and ``Lack of knowledge about how to interpret the data'', which relates directly to the topic of Data Literacy.

\section{Hypothesis}

In the light of open data theoretical benefits, and the impediments that hinder the achievement of these benefits, we formulate two hypothesis in order to guide the development of this thesis:

\noindent\textbf{\hypothesis{hyp:dataliteracy}:} Increasing the level of data literacy on the society leads to a more conscious use of open data;

\noindent\textbf{\hypothesis{hyp:tagging}:} Enhancing the organization of open data repositories leads to better find ability of open data.

Both hypothesis are inspired in the problems highlighted above.
In the first one, we assume that it is possible to increase the collective level of data literacy, and that this can lead to a deeper understanding of data.
Thus, we could gradually move from a scenario where data is always interpreted by intermediaries, to another where people can read, understand and even produce data themselves.

The second hypothesis brings to the light the data organization problem in the context of making data more easily available.
It assumes that, if by some means, data repositories could be better organized, data would be more easily accessible by a broader audience.

\section{Objective}

Our aim in this thesis is to develop approaches both from user and publisher perspectives in order to advance towards the benefits of open data. Specifically:
\begin{itemize}
	\item From the publisher perspective, and recognizing that the lack of organization of ODPs is a problem, to develop an approach for cleaning, reconciliation, and semantic lifting of metadata;
	\item From the user perspective, and recognizing the lack of abilities for dealing with data, to develop an approach for raising data literacy skills on communities of data users.
\end{itemize}

\section{Methodology}

The personal motivation for this work came from my previous experience in developing information systems for social movement activists, such as mapping systems, social networks and data analysis systems.
This practical experiencing showed that a recurrent error is to look only to strict technical problems, ignoring influences of the social context.
A learned lesson is that socio-technical problems need socio-technical approaches in order to be effective.
After attending to the Ph.D. seminars, the problem could understood, and a solution path could be envisioned.
This process resulted in the formulation of both hypothesis, that were necessary to enable tackling social and technical aspects.

The first methodological step after broadly defining the problem was a literature revision on open data, described in \autoref{chap:opendata}.
With this knowledge in hand, it was necessary to go to the field in order to experiment open data in practice.
Thus, theory about Popular Education and Participatory Research methods was reviewed, and the data literacy course was developed and applied, in a process reported in \autoref{chap:dataliteracy}.

As a result of this process, the problem definition and the solution approach were refined in order to fit the findings of the participatory research.
Thus, it was also necessary to refine the literature revision, and also to look at the reality of metadata usage in Open Data Portals.
This is the topic of \autoref{chap:relworks}.

With all these components in hand: (i) a broader literature review; (ii) a field research; (iii) a narrower literature review; and (iv) a reality analysis, it was possible to dive into the development of the solution, described in \autoref{chap:tagging}, and run its validation, described in \autoref{chap:evaluation}.
Finally, on the conclusions (\autoref{chap:evaluation}), we discuss in which extent our results can be generalized in face of the driven experiments, and if our hypothesis were fully validated.

\section{Structure of the thesis}

The remainder of this thesis is organized as follows:

\autoref{chap:opendata} presents an introduction to the main subject of this thesis, i.e., open data.
After an overview on the topic, we present some current research challenges in this field, with a special focus on the lack of people's capacity for dealing with data, and the lack of organization and linking possibilities on Open Government Data Portals. 
For some parts of this chapter, the work published in~\citeonline{Tygel2016} will be used.

\autoref{chap:dataliteracy} has a twofold objective: on the one hand, we present the results of a participatory research with open data users about their main motivations and impediments, and the wanted improvements on open data platforms. 
On the other hand, we systematise the research method into an open data course inspired in the principles of Popular Education.
The course methodology is presented, as well as some contributions on critical data literacy.
As a result of the research, the problem of linking and organizing data in ODPs appears as an outstanding impediment for using open data.
In this chapter, we will use the results published in~\citeonline{Tygel2015} and~\citeonline{Tygel2015a}.

\autoref{chap:relworks} goes deeper in analysing how previous research dealt with enhancement of metadata in ODPs.
A special focus is given on methods for extracting semantics of metadata, specially when dealing with tags.

\autoref{chap:tagging} presents the STODaP -- Semantic Tags for Open Data Portals -- approach.
The main purpose of this approach is to tackle the issue of OGD organization and linking, by cleaning up tags in ODPs, and creating a central repository for semantically annotated metadata.
The approach is composed by several strategies, both in the context of individual ODPs and between them.
In this section, we will benefit from the work published in~\citeonline{Tygel2016a}.

In \autoref{chap:evaluation} we drive an evaluation of the proposed approach.
A theoretical background about search engine evaluation is reviewed, and we present our methodology.
XX subjects compared the STODaP server with other search engines, both generic and data specialized ones.
Task completion time was calculated, and results show that subjects using the STODaP server were to complete tasks quicker than using other methods.
We discuss other dimensions of the results and to which extend we can generalize it.

\autoref{chap:conclusions} will present the concluding remarks of this thesis, highlighting our contributions, stating the limitations of our approach, and signalling ways for researchers willing to continue this work.
