\setlength{\absparsep}{18pt} % ajusta o espaçamento dos parágrafos do resumo
\begin{resumo}[Resumo]
A publicação extensiva de dados em formatos abertos na Internet parece ser uma tendência irreversível.
No caso de governos, pressões da sociedade por mais transparência vem fazendo com que cada vez mais, administrações públicas facilitem acessos aos dados de governo criando centrais únicas de acesso.
Estes repósitorios centrais de dados, ou portais de dados abertos, têm como objetivo ser um ponto de encontro entre o governo e cidadãos e cidadãs que desejam acessar dados públicos.
Espera-se com isso uma maior transparência das administrações públicas, que por sua vez alimenta a democracia com a população bem informada, e inibe desvios de recursos através da possibilidade de uma inspeção aberta ao público.
Junto às grandes expectativas geradas pelas políticas de dados abertos, verifica-se também uma ampla gama de problemas que ainda retardam o crescimento das iniciativas. 
No decorrer da pesquisa relacionada a esta tese, dois problemas chamaram a atenção: (i) a falta de organização conceitual dos portais de dados abertos e (ii) as dificuldades do público em geral em lidar com os dados.
Deste modo, esta tese busca trazer uma contribuição para o campo dos dados abertos propondo abordagens para ambos os problemas.
Administradores dos portais de dados abertos utilizam diversos tipos de metadados para organizar seus conjuntos de dados, sendo o mais importante as tags.
Entretanto, o processo de tageamento é sujeito a diversos problemas, como sinonimia, ambiguidade ou incoerência, entre outros. 
Como demonstrado por nossa análise empírica dos ODPs, estes problemas são atualmente prevalente na maior parte dos portais, e efetivamente dificultam o reuso dos dados abertos. 
Face a estes problemas, nesta tese foi desenvolvida e implemenatada uma abordagem para reconcialição de tags em portais de dados abertos, incluindo ações locais relacionadas individualmente aos portais, e ações globais para adição de uma camada de metadados semânticos acima dos portais individuais.
Diversos estudo apontam ainda que, dentro das políticas de dados abertos adotadas pelos governos, há um foco maior na publicação de dados, e uma menor atenção à capacidade de uso destes dados pelo público.
Assim, ainda que os dados sejam publicados, é necessário que haja um público capacitado para lidar eles.
Do contrário, corre-se o risco de criar uma elite capaz de tirar proveito destas informações, e aprofundar ainda mais a exclusão digital, sobretudo em países com o Brasil.
Neste sentido, apresentamos nesta tese uma abordagem para alfabetização em dados, inspirada na pedagodia da educação popular e na técnica de pequisa-ação participativa.
Deste modo, espera-se com essa tese contribuir com o avanço da democratização das informações, contextualizando de forma mais adequada a publicação de dados abertos, e permitindo um uso mais ampliado pela população.

\textbf{Palavras-chave}: Dados Abertos. Portal de Dados Abertos. Conciliação de Metadados. Enriquecimento Semântico. Alfabetização em Dados.
\end{resumo}

