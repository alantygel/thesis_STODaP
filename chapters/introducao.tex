\chapter{Introduction}

This thesis is essentially about open data.
Given the growing importance of the topic to the society, open data is discussed through several points of views.
Although looking at it from a Computer Science perspective, it was not possible to skip political, social and economical aspects while discussing the topic.
This work should thus be regarded as an effort to contribute to a multidisciplinary field that is heavily related to Computer Science, but far from being restricted to it.

In this introductory chapter, some motivations behind the topic of open data are briefly exposed, highlighting selected problems observed in the literature.
The hypothesis from where this thesis starts is posed, as well as the main and specific objectives that we aim to achieve with this work.
We further introduce our solution approach, and explain the methodology used in order to develop it. Finally, the structure of the remainder of this text is described.

\section{Motivation}
% * crescimento do open data
Current numbers about the open data scene leave no doubt about the central importance of this topic in contemporary society.
The Open Data Index\footnote{Available at \url{http://index.okfn.org/place/}.} monitored in 2015 open datasets published by 122 countries all over the world, on topics related to budget, national statistics, procurements, maps and many others.
Regarding the European landscape, the {Open Data Monitor\footnote{Available at \url{http://opendatamonitor.eu}.} counts 173 open data catalogues in the continent, which sums an amount of 1472 GB of data.

The movement towards opening datasets has its roots related to a series of access to information laws. 
According to the right2info.org\footnote{Available at \url{http://right2info.org}.} platform, these are laws that ``establish the right and procedures for the public to request and receive government-held information''.
\citeonline{Yannoukakou2014} drives a comprehensive description about the synergies between the right to information and open data movements.
According to them, both movements can push the formulation of an universal approach on access of government information.
\citeonline{Vleugels2012} presents a comprehensive list of 273 Freedom of Information Acts (FOIA), being 93 of national, 180 of sub-national and 3 of international scope.
Even though the first occurrence of this kind of law dates from 1766, in Sweden, the vast majority of them were created after the year 2000.

It is no coincidence that 9 years later, United States and United Kingdom launched their Open Data Portals (ODPs), a one-stop-shop for publishing and consuming government data.
Nowadays, several countries have already implemented their ODPs, together with numerous states and municipalities.
Universities and research centres are also joining strategies for putting data available on the Web.
More than 1600 ODPs were surveyed by OpenDataSoft\footnote{Available at \url{https://www.opendatasoft.com}.}.
The potential of changing the very basis of democratic processes took the United Nations (UN) to coin the term \emph{Data Revolution} to designate ``the new world of data, a world in which data are bigger, faster and more detailed than ever before''~\cite[p.4]{DataRevolutionGroup2014}.
Still according to the UN, these data is ``creating unprecedented possibilities for informing and transforming society and protecting the environment''~\cite[p.4]{DataRevolutionGroup2014}.

A study by \citeonline{Huijboom2011} comparing national open data strategies of five countries analysed their public policies programmes and key motivations behind publishing open data.
Authors defined three main categories of motivations:
\begin{itemize}
	\item Increase democratic control and political participation, i.e., open data could empower citizens on exercising their democratic rights;
	\item Foster service and product innovation, i.e., open data could generate new opportunities for innovation on the public and private sector. Under this topic, it has been recently stated that “Open data can help unlock U\$3 trillion to U\$5 trillion in economic value annually” \cite{Manyika2013}.
	\item Strengthen law enforcement, i.e., open data could enable citizen involvement and enable the development of security application.
\end{itemize}

While the big numbers and great expectations about open data may generate an enthusiastic hope that this movement will solve many problems of the society, there are also critique voices claiming that the promises are still far from being realised, and also that there are some hidden threats that should be alerted.
The problems of open data are the subject of next section.

\section{Not Only Advantages -- Open Data Impediments}

In front of such a big hope regarding the benefits of opening datasets, several authors have also dedicated themselves to analyse open data from a critical point of view \cite{Zuiderwijk2012, Zuiderwijk2014, Gurstein2011, Bates2014, Roseira2016, Parycek2016, Davies2012b}
\footnote{Open data problems will be deeper analysed in \autoref{sec:problems}.
For the purposes of this Introduction, only the most relevant to this work will be detailed.}.

Among them, it is possible to divide open data criticism in two categories: (i) \emph{Problems}, i.e., technological and political implementation failures that prevent open data to achieve their desired goals, and (ii) \emph{Perils}, i.e., unexpected outcomes of open data implementation that are negative for the society as a whole, or specific groups.

Regarding the Problems, \citeonline{Zuiderwijk2012}, in a very complete work, collected 118 socio-technical impediments for use of open data from interviews, workshops and literature.
Some cited impediments were ``absence of commonly agreed metadata'', ``insufficiency of metadata'', ``the lack of interoperability'' and ``difficulty in searching and browsing data'', showing that data organization is great challenge for open data.

The open data organization challenge can be subdivided into two aspects: 1) structuring and organizing the datasets themselves and 2) providing well-structured and organized metadata for the datasets.

The first aspect has been recently approached using the Linked Open Data (LOD) paradigm \cite{Berners-Lee2006}.
In short, this approach aims to semantically enrich data by giving unique identifiers (Uniform Resource Identifier - URIs) to elements of a dataset, and linking these identifiers to commonly agreed knowledge bases, or web ontologies.
The process of connecting isolated datasets to the LOD cloud is called semantic lifting, and was applied to open datasets by~\citeonline{DBLP:conf/i-semantics/ErmilovAS13} and~\citeonline{Ding2011325}, who tried to build general strategies for putting large open government datasets in the Link Open Data (LOD) cloud.

Regarding the second aspect, the Data Catalog Vocabulary (DCAT)\footnote{Available at \url{http://www.w3.org/TR/vocab-dcat/}}~\cite{conf/i-semantics/CyganiakMP10} was developed to provide standardized metadata for open data catalogues.
If DCAT succeeds in providing a standard structure for describing open datasets, e.g., defining fields such as \texttt{dcat:title}, \texttt{dcat:description}, \texttt{dcat:keyword} and \texttt{dcat:theme}, harmonizing the content of these fields is out its scope.
This means that, if we have for Dataset 1 \texttt{dcat:theme spending plan}, and for Dataset 2 \texttt{dcat:theme budget}, both are not linked, although they are dealing with the same subject.


Besides the data organization challenge, another commonly cited impediment to the use of open data is the individuals and groups' lack of capacity for dealing with open data.
There has been a recently growing consensus on defining the skills of consuming, publishing and understanding data under the concept of \emph{Data Literacy}.

As observed by~\citeonline{Bhargava2015}, one of the first mentions to the term \emph{Data Literacy} called the attention for its importance on the context of evaluation of information, together with  Information Literacy and Statistical Literacy. 
In 2004, Schield reinforced the importance of teaching these three literacies for ``students who need to critically evaluate information in arguments'' \cite[p.1]{Schield2004}.

Although not directly mentioning  the term Data Literacy, the previously cited collection of open data impediments~\cite{Zuiderwijk2012} dedicates a section for problems related to \emph{understand ability}.
Among them we find, for example, ``Lack of skills and capabilities to use the data and'' and ``Lack of knowledge about how to interpret the data'', which relate directly to the topic of Data Literacy.

\section{Hypothesis}

In the light of open data theoretical benefits, and the impediments and perils that hinder the achievement of these benefits, we formulate a hypothesis in order to guide the development of this thesis:

% \noindent\textbf{\hypothesis{hyp:dataliteracy}:} ; >> motivação

% confirme que abordagem de soluçao está correta, mas foi mais exploração do problema

\noindent\textbf{\hypothesis{hyp:tagging}:} \emph{Enhancing the organization of open data repositories leads to better find ability of open data.}

% Both hypothesis are inspired in the problems highlighted above.

The hypothesis puts in evidence the data organization problem with the aim of making data more easily available.
It assumes that, if by some means, data repositories could be better organized, data would be more easily accessible by a broader audience.

The first part of H1 -- \emph{enhancing the organization of open data repositories} -- assumes that open data repositories, or ODPs, exist, and that they are not organized.
Although not exclusively, ODPs are mostly run by governments willing to publish public data, which are also called Open Government Data (OGD).
Our aim is not to develop an exclusive approach for the government context.
However, because most of data available is related to OGD, a bias on this direction can be introduced.
Thus, the existence of ODPs is clear from the first section of this Introduction, and organization problems will be clarified during this text, specifically in \autoref{sec:problems} and \autoref{chap:relworks}.

The second part of H1 -- \emph{better find ability of open data} -- is related to people wanting to access data.
The first assumption is that data consumers exist, i.e., that there is a demand from the society for open data.
Several works point out the missing focus on users, or data consumers.
Topics regarding open data demand and motivations are detailed in the next chapter, but it is worth mentioning here the work by \citeonline{Davies2012}. 
It describes a survey about the different uses of open data, and gives a more realistic perspective on how people really use data.
The second assumption is that people currently have problems in finding those data.
Reports of problems on finding open data are also available on the literature.
\citeonline{Zuiderwijk2012} summarizes 6 categories of find ability impediments, collected from the literature, workshops and interviews.

Hypothesis H1 goes only until the point when data is found.
But what happens after users get access to the desired open dataset?

Solving the organization problem of open data will not make it more accessible alone.
Increasing the level of data literacy on the society is also of paramount importance in order to democratize access to data, and their claimed benefits.
Otherwise, there is a serious risk of creating a ``Data Divide'', i.e., extending the inequalities of our society to the open data access and thus \emph{empowering the empowered}, as stated by \citeonline{Gurstein2011}.
Although not formulated as a hypothesis, mainly because of our lack of instruments to validate it, we assume in this thesis that an increase on the collective level of data literacy could lead to a democratisation of access and benefits of open data.
Thus, as a motivation of our work, we understand and assume the importance of gradually moving from a current scenario where the society is exposed to data filtered and explained by intermediaries, to another where people can critically read, understand and even produce data themselves.

\section{Objectives}
\label{sec:objectives}
Our aim in this thesis is to develop an approach to enhance the organization of open data repositories, with the perspective of facilitating access to data, and consequently improving the realisation of open data benefits in a democratic way.
In order to accomplish this main target, we formulate as specific objectives (SO):
\begin{enumerate}
	\item To provide an overview on the state of the art regarding the open data landscape;
	\item To investigate in which extent Data Literacy efforts can contribute to democratisation of access to open data;
	\item To systematise literature efforts on semantically enhancing open data descriptors;
	\item To analyse the state of the practice on metadata usage on the ODP context;
	\item To develop an approach for helping ODP managers enhance their data organization;
	\item To develop an approach for semantically connecting ODPs, enabling an integrated and meaningful data search
	% \item To develop an approach for raising data literacy skills on communities of data users.
\end{enumerate}

In the following, the approach for reaching those targets is explained.

\section{Solution Approach}

In order to validate the proposed hypothesis, and to accomplish our main objective, we propose the Semantic Tags for Open Data (STODaP) approach.
As detailed in \autoref{chap:tagging}, the STODaP approach consists in merging local strategies (at the ODP level) for enhancing metadata quality, and global strategies, developing a common environment for accessing open data.

Open Data Portals are collections of datasets, which have a set of metadata to describe it.
A widely used type of metadata are tags, free-text labels that can describe topics, geographical regions, temporal informations or other aspects regarding datasets.

Our approach consists in semantically lifting these tags so that datasets from different portals, eventually in different languages, can be connected if their content refers to the same subject. 
Using the STODaP server, users can find data about specific topics from several sources in one place.

In order to specify and develop this main objective, SOs have necessarily to be accomplished.
SO1 to SO4 are aimed at understanding our problem from its several perspectives various, respectively the general open data landscape, the voice of (potential) open data users, related works specific about open data organization, and the use actual use of metadata in ODPs.
SO5 and SO6 are subdivisions of the main product of this thesis - the STODaP approach.



\section{Methodology}

The personal motivation for this work came from my previous experience in developing information systems for social movement activists, such as mapping systems, social networks and data analysis systems.
This practical experience showed that a recurrent error is to look only to strict technical problems, ignoring influences of the social context.
A learned lesson is that socio-technical problems need socio-technical approaches in order to be effective.
After attending to the Ph.D. seminars, the problem could be understood, and a solution path could be envisioned.
This process resulted in the formulation of both hypothesis, that were necessary to enable tackling social and technical aspects.

The first methodological step after broadly defining the problem was a literature revision on open data, described in \autoref{chap:opendata}.
With this knowledge in hand, it was necessary to go to the field in order to experiment open data in practice.
Thus, theory about Popular Education and Participatory Research methods was reviewed, and the data literacy course was developed and applied, in a process reported in \autoref{chap:dataliteracy}.

As a result of this process, the problem definition and the solution approach were refined in order to fit the findings of the participatory research.
Thus, it was also necessary to refine the literature revision, and also to look at the reality of metadata usage in Open Data Portals.
This is the topic of \autoref{chap:relworks}.

With all these components in hand: (i) a broader literature review; (ii) a field research; (iii) a narrower and deeper literature review; and (iv) a reality analysis, it was possible to dive into the development of the solution, described in \autoref{chap:tagging}, and conduct a validation, described in \autoref{chap:evaluation}.
Finally, on the conclusions (\autoref{chap:evaluation}), we discuss in which extent our results can be generalized in face of the driven experiments, and if our hypothesis were fully validated.

\section{Structure of the thesis}

The remainder of this thesis is organized as follows:

\autoref{chap:opendata} presents an introduction to the main subject of this thesis, i.e., open data.
After an overview on the topic, we present some current research challenges in this field, with a special focus on the lack of people's capacity for dealing with data, and the lack of organization and linking possibilities on Open Government Data Portals. 
For some parts of this chapter, the work published in~\citeonline{Tygel2016} will be used.

\autoref{chap:dataliteracy} has a twofold objective: on the one hand, we present the results of a participatory research with open data users about their main motivations and impediments, and the wanted improvements on open data platforms. 
On the other hand, we systematise the research method into an open data course inspired in the principles of Popular Education.
The course methodology is presented, as well as some contributions on critical data literacy.
As a result of the research, the problem of linking and organizing data in ODPs appears as an outstanding impediment for using open data.
In this chapter, we will use the results published in~\citeonline{Tygel2015} and~\citeonline{Tygel2015a}.

\autoref{chap:relworks} sets the theoretical and practical basis to our solution.
It goes deeper in analysing how previous research dealt with enhancement of metadata in ODPs.
A special focus is given on methods for extracting semantics of metadata, specially when dealing with tags.
We also drive an analysis of metadata usage in open data portals.

\autoref{chap:tagging} presents the STODaP -- Semantic Tags for Open Data Portals -- approach.
The main purpose of this approach is to tackle the issue of OGD organization and linking, by cleaning up tags in ODPs, and creating a central repository for semantically annotated metadata.
The approach is composed by several strategies, both in the context of individual ODPs and between them.
In this chapter, we will benefit from the work published in~\citeonline{Tygel2016a}.

In \autoref{chap:evaluation} we drive an evaluation of the proposed approach.
A theoretical background about search engine evaluation is reviewed, and we present our methodology.
34 subjects compared the STODaP server with other search engines, both generic and data specialized ones.
Task completion time was calculated, and results show that subjects using the STODaP server were to complete tasks quicker than using other methods.
% We discuss other dimensions of the results and to which extend we can generalize it.

\autoref{chap:conclusions} will present the concluding remarks of this thesis, highlighting our contributions, stating the limitations of our approach, and signalling ways for researchers willing to continue this work.
